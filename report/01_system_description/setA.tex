\chapter{Requirement Set A}
\label{ch:setA}

\section{Process Handling}
\label{se:process_handling}

\paragraph*{PH1: Process States}
\addcontentsline{toc}{paragraph}{PH1: Process States}
\label{pg:ph1}

\begin{itemize}
  \item Processes should be possible to block.
  \item  A blocked process should not be able to run until it is unblocked.
  \item  It should be possible to delay a process for a specified amount of time.
  \item  When a process is delayed, it should not be able to execute any code.
\end{itemize}

\paragraph*{PH2: Process Priorities}
\addcontentsline{toc}{paragraph}{PH2: Process Priorities}
\label{pg:ph2}

\begin{itemize}
  \item Each process should have a priority.
  \item  The number of priority levels should be at least 30.
  \item  Priorities should be specified at process start.
  \item  Priorities should be changeable at run-time.
\end{itemize}

\paragraph*{PH3: Process Scheduling}
\addcontentsline{toc}{paragraph}{PH3: Process Scheduling}
\label{pg:ph3}

Scheduling should be highest-priority first, with preemption. In particular:

\begin{itemize}
  \item  A higher-priority process should be scheduled in preference to a lower-priority process.
  \item  Between processes of the same priority, round-robin scheduling should be used.
  \item  Higher-priority processes should be able to preempt lower-priority processes.
  \item  There should be a recurring regular check for which process to run, unrelated to whether a particular task has finished its work.
  \item  A lower priority process should not be allowed to run if there is a higher priority process ready to run. We expect strict highest-priority-first scheduling.
\end{itemize}

\paragraph*{PH4: Process Identities}
\addcontentsline{toc}{paragraph}{PH4: Process Identities}
\label{pg:ph4}

\begin{itemize}
  \item Each process should have a unique ID that can be used to address the process and perform operations on it.
  \item  A process may have a human-readable name.
\end{itemize}

\paragraph*{PH5: Process Creation}
\addcontentsline{toc}{paragraph}{PH5: Process Creation}
\label{pg:ph5}

\begin{itemize}
  \item Processes should be able to create new processes.
  \item  Processes are created by a spawn operation that starts a new process with a given code to run (see requirement PH7 p.\pageref{pg:ph7}).
  \item  The spawn operation should return the ID of the created process.
  \item  If the process spawning fails for some reason, an error should be reported.
\end{itemize}

\paragraph*{PH6: Process Termination}
\addcontentsline{toc}{paragraph}{PH6: Process Termination}
\label{pg:ph6}

\begin{itemize}
  \item Processes should be able to terminate themselves.
  \item  Processes should be able to terminate other processes.
  \item  The operating system should be able to terminate processes.
\end{itemize}

\paragraph*{PH7: Programs and Processes}
\addcontentsline{toc}{paragraph}{PH7: Programs and Processes}
\label{pg:ph7}

\begin{itemize}
  \item Each process runs the code of a particular program.
  \item  The program to execute in a process is specified when the process starts.
  \item  Each program should have an ID that is used to identify it when starting a process.
  \item  Each program should have a human-readable name associated with it. This might be the same as the ID.
  \item  There should be a global table of available programs.
  \item  It should be possible to start every program in multiple simultaneous instances.
\end{itemize}

\paragraph*{PH8: Process Information}
\addcontentsline{toc}{paragraph}{PH8: Process Information}
\label{pg:ph8}

\begin{itemize}
  \item A process should be able to obtain information about another process.
  \item  The information should include:
	 \begin{itemize}
		\item  Process priority
		\item  Process name, if such is used
		\item  Name of the program running in the process
		\item  Current scheduling state of the process
	 \end{itemize}
  \item  This information should be readable by the process requesting it.
  \item  The information should not be printed directly to the console; if it is to be presented to the user, a program (or function in the shell program) will have to be written for that purpose.
\end{itemize}

\paragraph*{PH9: Resource Recycling}
\addcontentsline{toc}{paragraph}{PH9: Resource Recycling}
\label{pg:ph8}

\begin{itemize}
  \item When a process is terminated, all resources that it has allocated should be reclaimed.
  \item  Process IDs should be recycled.
  \item  The operating system should never ``leak'' memory, if there are dynamically allocated or assigned resources.
\end{itemize}

\paragraph*{PH10: Process Limitations}
\addcontentsline{toc}{paragraph}{PH10: Process Limitations}
\label{pg:ph10}

\begin{itemize}
  \item There can be a fixed upper limit on the number of processes allowed in the  system.
  \item  Process IDs should be reused, as per requirement PH9 p.\pageref{pg:ph9}.
  \item  Any fixed upper limits must be easy to modify, using a \#define o or similar C construct.
\end{itemize}

\section{Input and Output}
\label{se:input_output}

\paragraph*{IO1: Text Consoles}
\addcontentsline{toc}{paragraph}{IO1: Text Consoles}
\label{pg:io1}

\begin{itemize}
  \item The system should provide at least one text console for input and output.
\end{itemize}

\paragraph*{IO2: Interrupt-Driven IO}
\addcontentsline{toc}{paragraph}{IO2: Interrupt-Driven IO}
\label{pg:io2}

\begin{itemize}
  \item Text input and output shall be implemented using interrupt-driven IO on the serial port of the system.
\end{itemize}

\paragraph*{IO3: Text Output}
\addcontentsline{toc}{paragraph}{IO3: Text Output}
\label{pg:io3}

\begin{itemize}
  \item It should be possible for processes to send text to a console.
  \item  It should be possible to send coherent messages, i.e. a process should be able to send a text message that is printed uninterrupted on a text console.
\end{itemize}

\paragraph*{IO4: Text Input}
\addcontentsline{toc}{paragraph}{IO4: Text Input}
\label{pg:io4}

\begin{itemize}
  \item It should be possible for a process to request the user to input text.
  \item  Text input should be returned as a string to the requesting process.
\end{itemize}

\paragraph*{IO5: Malta LCD Output}
\addcontentsline{toc}{paragraph}{IO5: Malta LCD Output}
\label{pg:io5}

\begin{itemize}
  \item It should be possible for a process to output text to the Malta LCD display
\end{itemize}

\section{System Application Programming Interface (API)}

\paragraph*{A1: General}
\addcontentsline{toc}{paragraph}{A1: General}
\label{pg:a1}

\begin{itemize}
  \item The API should be the only way to request services from the operating system kernel.
\end{itemize}

\paragraph*{A2: Syscall}
\addcontentsline{toc}{paragraph}{A2: Syscall}

\begin{itemize}
  \item The MIPS Syscall exception mechanism should be used to implemented the API.
\end{itemize}

\paragraph*{A3: Documentation}
\addcontentsline{toc}{paragraph}{A3: Documentation}

\begin{itemize}
  \item The API should be presented to the user programs as a set of C functions.
  \item  The API should be documented for the benefit of user programs.
\end{itemize}

\section{Other}

\paragraph*{O1: No Shared Memory}
\addcontentsline{toc}{paragraph}{O1: No Shared Memory}

\begin{itemize}
  \item Processes should never use shared variables or memory to communicate.
  \item  It is not necessary for the operating system to rigorously enforce this condition.
\end{itemize}

\paragraph*{O2: Reentrancy}
\addcontentsline{toc}{paragraph}{O2: Reentrancy}

It should be possible to execute multiple instances of the same program concurrently. In particular, this means that:

\begin{itemize}
  \item A program cannot use ``static'' variables in C or global variables.
  \item  Any attempts at using names to identify processes has to allow for the dynamic creation of names.
\end{itemize}

\paragraph*{O3: Error Handling}
\addcontentsline{toc}{paragraph}{O3: Error Handling}
Whenever the operating system cannot complete an operation, it should return an error code to the process trying to perform the operation.

\begin{itemize}
  \item Error codes should be returned to the calling process as the function result, so that it has the ability to deal with the error.
  \item  To help debugging programs, the OS may provide the facility to print diagnostic errors to the console automatically.
\end{itemize}

\section{User Programs}

The system would be no fun without a set of programs to run on it. For demonstration purposes, you should implement a set of simple user programs.

\paragraph*{UP1: Increment}
\addcontentsline{toc}{paragraph}{UP1: Increment}
number sequence 1, 2, 3, \ldots., N to the console. Each number should be printed on a separate line.

\paragraph*{UP2: Fibonacci}
\addcontentsline{toc}{paragraph}{UP2: Fibonacci}

A user level program that prints out the Fibonacci number serie: 1, 1, 2, 3, 5, 8, 11, \ldots\ldots. Max-Fib-Number. Each number should be printed on a separate line.

\paragraph*{UP3: Command Shell}
\addcontentsline{toc}{paragraph}{UP3: Command Shell}

The system should have a command shell that can do at least the following:

\begin{itemize}
  \item Start processes. For example, it shall be possible to start another interpreter.
  \item  Change priority of processes.
  \item  Obtain information about present processes.
  \item  Terminate processes.
  \item  Ouput to the Malta LCD display
  \item  It should be possible to have more than one shell running simultaneously.
  \item  The command shell should not be special-treated by the operating system in any way.
  \item  The command shell has to handle ``backspace'' functionality when entering commands.
  \item  The command shell should not use CPU time when waiting for input.
\end{itemize}

\paragraph*{UP4: Text Scroller}
\addcontentsline{toc}{paragraph}{UP4: Text Scroller}

There should be a scroller background process that scrolls text on the Malta board LCD display.

\begin{itemize}
  \item The text can be varied or fixed, depending on the level of ambition.
  \item  The process should provide for smooth scrolling even on a highly loaded system.
  \item  The scroller should be a regular user process with high priority.
  \item  The scroller should sleep between updates to the display.
  \item  The scroller should start when the operating system starts.
\end{itemize}

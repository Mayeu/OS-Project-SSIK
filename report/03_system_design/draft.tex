\chapter{Kernel Structure}

\section{Kernel}

link all other module
start the system

contain the different process list (ready, running, waiting, terminated)

\section{Process module}

\subsection{kprocess.c}

manage process individualy

error create( program\_name, pcb )
error delete( pid )
error modify\_something( pid, other info ) a bunch of functions to modify the PCB
error read\_info (pid, pcbinfo* ) fill a pcbinfo  struct with pcb information. This disallow pcb modification by anybody
error copy ( pcbsrc*, pcbdest* )

\subsection{kprocess\_list.c}

manage a list of process

struct pls pcb[defined\_size]

error create( pls* ) 
error delete ( pls* )
pcb* empty\_space( pls* ) return the first empty position in the list as a pointer
bool is\_empty ( pcb* ) is this pcb position empty ?
pcb* search ( pid, pls* ) search a pid in a process list
error move (pid, plssrc*, plsdest* ) move a process from a list to another (will search to ensure that the pcb is in the list
error sort (pls*) sort a process list by priority (highest to lowest)

\section{Scheduler module}

work in 2 mode
- highest priority
- round robin for equal priority

long term scheduling happen when a process terminate or is added to the ready list or go to the waiting list.
It search for the highest pri among the ready queue and move it to the running queue. If this process have the priority on the process in the running list, all the current running process are moved back to the waiting list.

short term scheduling happen every Xms by a timer interuption. This switch the context to the next running process.

\section{Error module}

module for error, interruption, exception handling
provide facility for printing error

\subsection{kerror.c}

provide functions to print error to set the error pointer and everything error related

\subsection{kexception.c}

exception handler

\section{System Library}

\subsection{string.c}

strcpy
strcpyn
strcmp
strcmpn

\subsection{stdio.c}

printf
getc

\subsection{error.h}

provide all the error code

\section{Kernel Library}

this module implement all the function in System Library that need a Syscall

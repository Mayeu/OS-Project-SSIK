\section{Message}
The aim of this module is to allow processes to communicate between each
others asynchronously.
It provides some functions to send and receive a message, and a set of
functions to manage a list of messages.

This module is dependent of the module Process in the sense that it need that
two processes are created and know the pid of the other to send/receive messages.

To manage the messages, we have two structures:
\begin{description}
\item[msg] which represents a message. The structure is made of:
	\begin{itemize}
		\item{int msg\_id} the message identifier
		\item{int pid\_send} the process identifier of the sender
		\item{int pid\_recv} the process identifier of the receiver
		\item{int pri} the priority of the message
		\item{void *user\_data} some user data included into the message
	\end{itemize}
\item[msg\_lst] which represents the list of msg received in each process.
				It is made of:
	\begin{itemize}
	\item{msg *msg} a message
	\item{msg\_lst *msg\_lst} a pointer to the next element in the list
	\end{itemize}
				
\end{description}


\subsection{Message functions (kmsg.c)}

\subsubsection{create\_msg}
\begin{itemize}
\item{Description}

	Create a new message object with the given information and send it to
	the receiver (i.e. add it to the message list of the receiver).
\item{Parameters}
	\begin{description}
	\item[int msg\_id] the message identifier
	\item[int pid\_send] the process identifier of the sender
	\item[int pid\_recv] the process identifier of the receiver
	\item[int pri] the priority of the message
	\item[void *user\_data] some user data included into the message
	\end{description}
\item{Return}
	\begin{description}
	\item[int] the error identifier in case of any failure
	\end{description}
\end{itemize}

\subsubsection{receive\_msg}
\begin{itemize}
\item{Description}

	Wait for a message with a priority equals to \textit{pri} from any
	process. Delete the message after received it.
\item{Parameters}
	\begin{description}
	\item[int timeout] the maximum time a process is waiting for a message. 0 means infinite time.
	\item[int pri] the priority of the message to receive
	\item[msg*] the received message. null value if no such
		message has been received.
	\end{description}
\item{Return}
	\begin{description}
	\item[int] the error identifier in case of any failure
	\end{description}
\end{itemize}

\subsubsection{delete\_msg}
\begin{itemize}
\item{Description}

	Delete the message identified by msg\_id from the message list.
\item{Parameters}
	\begin{description}
	\item[int msg\_id] the identifier of the message
	\end{description}
\item{Return}
	\begin{description}
	\item[int] the error identifier in case of any failure
	\end{description}
\end{itemize}


\subsection{Message list functions (kmsg\_lst.c)}

The following list functions are dependent of the message structure and functions.

\subsubsection{create\_msg\_lst}
\begin{itemize}
\item{Description}

	Create a new message list with no element.
\item{Parameters}
	\begin{description}
	\item[msg\_lst*] the new empty list
	\end{description}
\item{Return}
	\begin{description}
	\item[void]
	\end{description}
\end{itemize}

\subsubsection{add\_to\_msg\_lst}
\begin{itemize}
\item{Description}

	Add the message identified by msg\_id to the end of the list lst.
\item{Parameters}
	\begin{description}
	\item[msg\_lst* lst] the message list
	\item[int msg\_id] the message identifier
	\item[msg\_lst*] the initial list plus the new message
	\end{description}
\item{Return}
	\begin{description}
	\item[int] the error identifier in case of any failure
	\end{description}
\end{itemize}

\subsubsection{rm\_from\_msg\_lst}
\begin{itemize}
\item{Description}

	Remove the message identified by msg\_id from the list lst.
\item{Parameters}
	\begin{description}
	\item[msg\_lst* lst] the message list
	\item[int msg\_id] the message identifier
	\item[msg\_lst*] the initial list minus the message specified in parameter
	\end{description}
\item{Return}
	\begin{description}
	\item[int] the error identifier in case of any failure
	\end{description}
\end{itemize}

\subsubsection{rm\_msg\_lst}
\begin{itemize}
\item{Description}

	Remove all elements in the list and delete the list.
\item{Parameters}
	\begin{description}
	\item[msg\_lst* lst] the message list
	\end{description}
\item{Return}
	\begin{description}
	\item[int] the error identifier in case of any failure
	\end{description}
\end{itemize}

\subsubsection{lookup\_into\_msg\_lst}
\begin{itemize}
\item{Description}

	Lookup the message identified by msg\_id into the list lst.
\item{Parameters}
	\begin{description}
	\item[msg\_lst* lst] the message list
	\item[int msg\_id] the message identifier
	\end{description}
\item{Return}
	\begin{description}
	\item[bool] true if the message is found in the list, false otherwise
	\end{description}
\end{itemize}

\subsubsection{sort\_msg\_lst}
\begin{itemize}
\item{Description}

	Sort the message list according to the priority (highest first).
\item{Parameters}
	\begin{description}
	\item[msg\_lst* lst] the message list
	\item[msg\_lst* lst] the sorted message list
	\end{description}
\item{Return}
	\begin{description}
	\item[int] the error identifier in case of any failure
	\end{description}
\end{itemize}

\subsubsection{empty\_space\_msg\_lst}
\begin{itemize}
\item{Description}

	Find the fist empty space in the array representing the list.
\item{Parameters}
	\begin{description}
	\item[msg\_lst* lst] the message list
	\item[msg\_lst* lst] the message list
	\end{description}
\item{Return}
	\begin{description}
	\item[int] the error identifier in case of any failure
	\end{description}
\end{itemize}


\subsubsection{is\_empty\_msg\_lst}
\begin{itemize}
\item{Description}

	Specifies if the given list is empty or not.
\item{Parameters}
	\begin{description}
	\item[msg\_lst* lst] the message list
	\end{description}
\item{Return}
	\begin{description}
	\item[bool] true if the list is empty, false otherwise
	\end{description}
\end{itemize}


\section{Error}

The aim of this module is to help the developer to diagnostic what occured in
case of any failure.
\subsection{Functions (kerror.c)}

\subsubsection{print\_error}
\begin{itemize}
\item{Description}

	Print the specified err\_msg followed by the description of the
	error according to the global variable error\_no.
\item{Parameters}
	\begin{description}
	\item[string err\_msg] the message the user wants to add to the error message
	\end{description}
\item{Return}
	\begin{description}
	\item[void]
	\end{description}
\end{itemize}

In this module, we also define:
\begin{description}
\item[int *err\_no] Value of the last error that occured
\end{description}

\section{System library}

This module is dependent of the kernel library module beacause the following
functions only perform syscalls to the kernel functions.

\subsection{String functions (string.c)}

\subsubsection{strcpy}
\begin{itemize}
\item{Description}

	Copy the string src to dest.
\item{Parameters}
	\begin{description}
	\item[char *src] the source string
	\item[char *dest] the destination string
	\end{description}
\item{Return}
	\begin{description}
	\item[int] the error identifier in case of any failure
	\end{description}
\end{itemize}

\subsubsection{strcpyn}
\begin{itemize}
\item{Description}

	Copy the length first characters of the string src to dest.
\item{Parameters}
	\begin{description}
	\item[char *src] the source string
	\item[char *dest] the destination string
	\item[int length] the number of characters to copy
	\end{description}
\item{Return}
	\begin{description}
	\item[int] the error identifier in case of any failure
	\end{description}
\end{itemize}

\subsubsection{strcmp}
\begin{itemize}
\item{Description}

	Compare the two string str1 and str2 to specify if str1 = str2
	of which one of them is the first alphabetically.
\item{Parameters}
	\begin{description}
	\item[char *str1] the first string
	\item[char *str2] the second string
	\end{description}
\item{Return}
	\begin{description}
	\item[int] 0 means str1 = str2, -1 means str1 < str2 and 1 means
		that str1 > str2
	\end{description}
\end{itemize}

\subsubsection{strcmpn}
\begin{itemize}
\item{Description}

	Compare the first n characters of the two string str1 and str2 to
	specify if str1 = str2 of which one of them is the first alphabetically.
\item{Parameters}
	\begin{description}
	\item[char *str1] the first string
	\item[char *str2] the second string
	\item[int n] the number of characters to compare
	\end{description}
\item{Return}
	\begin{description}
	\item[int] 0 means str1 = str2, -1 means str1 < str2 and 1 means that
		str1 > str2
	\end{description}
\end{itemize}

\subsubsection{strlen}
\begin{itemize}
\item{Description}

	Specify the number of characters of the string str
\item{Parameters}
	\begin{description}
	\item[char *str] the string
	\end{description}
\item{Return}
	\begin{description}
	\item[int]	the length of the string. -1 if str invalid.
	\end{description}
\end{itemize}

\subsubsection{strchr}
\begin{itemize}
\item{Description}

	res is a pointer to the first occurrence of character c in the string str.
\item{Parameters}
	\begin{description}
	\item[char *str] the string
	\item[char c] the character to find
	\item[char *res] the substring (result)
	\end{description}
\item{Return}
	\begin{description}
	\item[int] the error identifier in case of any failure
	\end{description}
\end{itemize}

\subsubsection{isspace}
\begin{itemize}
\item{Description}

	Checks if parameter c is a white-space character (SPC, TAB, LF, VT, FF, CR).
\item{Parameters}
	\begin{description}
	\item[char c] the character to evaluate
	\end{description}
\item{Return}
	\begin{description}
	\item[bool] true means that c is a space character, false otherwise
	\end{description}
\end{itemize}

\subsection{Display functions (stdio.c)}

\subsubsection{printf}
\begin{itemize}
\item{Description}

	Print the string str to the standard output.
\item{Parameters}
	\begin{description}
	\item[char *str] the string to print
	\end{description}
\item{Return}
	\begin{description}
	\item[int] the error identifier in case of any failure
	\end{description}
\end{itemize}

\subsubsection{fprintf}
\begin{itemize}
\item{Description}

	Print the string str to the specified output.
\item{Parameters}
	\begin{description}
	\item[int out] the output where to print (0 = console, 1 = Malta)
	\item[char *str] the string to print
	\end{description}
\item{Return}
	\begin{description}
	\item[int] the error identifier in case of any failure
	\end{description}
\end{itemize}

\subsubsection{getc}
\begin{itemize}
\item{Description}

	Returns the character currently pointed by the internal file position indicator of the input stream (keyboard).
\item{Parameters}
	\begin{description}
	\item[void] 
	\end{description}
\item{Return}
	\begin{description}
	\item[char] the character read
	\end{description}
\end{itemize}

\subsubsection{fgets}
\begin{itemize}
\item{Description}

	Reads characters from stream and stores them as a string into str until (num-1)
	characters have been read or either a newline or a the End-of-File is reached,
	whichever comes first. stream (keyboard). A null character is added to the end.
\item{Parameters}
	\begin{description}
	\item[char *str]  the string read from the input stream (keyboard)
	\item[int num]  the number of charaters to be read
	\end{description}
\item{Return}
	\begin{description}
	\item[int] the error identifier in case of any failure
	\end{description}
\end{itemize}

\subsection{Error codes (error.h)}
	\begin{description}
	\item[SUCCESS] No error occured
	\item[OUTOMEM] Out of memory
	\item[UNKNPID] Unknown pid (process identifier)
	\item[UNKNMID] Unknown mid (message identifier)
	\item[INVPRI] Invalid priority
	\item[OUTOPID] Out of pid (number of processes is limited) 
	\item[OUTOMID] Out of mid (number of messages is limited) 
	\item[NULLPTR] Null pointer error
	\item[INVEID] Invalid eid (error identifier)
	\end{description}

\section{Kernel library}
\subsection{Functions (klib.c)}

\subsubsection{kprintf}
\begin{itemize}
\item{Description}

	Print the string str to the standard output.
\item{Parameters}
	\begin{description}
	\item[char *str] the string to print
	\end{description}
\item{Return}
	\begin{description}
	\item[int] the error identifier in case of any failure
	\end{description}
\end{itemize}

\subsubsection{kfprintf}
\begin{itemize}
\item{Description}

	Print the string str to the specified output.
\item{Parameters}
	\begin{description}
	\item[int out] the output where to print (0 = console, 1 = Malta)
	\item[char *str] the string to print
	\end{description}
\item{Return}
	\begin{description}
	\item[int] the error identifier in case of any failure
	\end{description}
\end{itemize}

\subsubsection{kgetc}
\begin{itemize}
\item{Description}

	Returns the character currently pointed by the internal file position indicator of the input stream (keyboard).
\item{Parameters}
	\begin{description}
	\item[void] 
	\end{description}
\item{Return}
	\begin{description}
	\item[char] the character read
	\end{description}
\end{itemize}

\subsubsection{kfgets}
\begin{itemize}
\item{Description}

	Reads characters from stream and stores them as a string into str until (num-1)
	characters have been read or either a newline or a the End-of-File is reached,
	whichever comes first. stream (keyboard). A null character is added to the end.
\item{Parameters}
	\begin{description}
	\item[char *str]  the string read from the input stream (keyboard)
	\item[int num]  the number of charaters to be read
	\end{description}
\item{Return}
	\begin{description}
	\item[int] the error identifier in case of any failure
	\end{description}
\end{itemize}

\subsubsection{kisspace}
\begin{itemize}
\item{Description}

	Checks if parameter c is a white-space character (SPC, TAB, LF, VT, FF, CR).
\item{Parameters}
	\begin{description}
	\item[char c] the character to evaluate
	\end{description}
\item{Return}
	\begin{description}
	\item[bool] true means that c is a space character, false otherwise
	\end{description}
\end{itemize}

\subsubsection{kstrchr}
\begin{itemize}
\item{Description}

	Update res which is a pointer to the first occurrence of character c in the string str.
\item{Parameters}
	\begin{description}
	\item[char *str] the string
	\item[char c] the character to find
	\item[char *res] the substring (result)
	\end{description}
\item{Return}
	\begin{description}
	\item[int] the error identifier in case of any failure
	\end{description}
\end{itemize}

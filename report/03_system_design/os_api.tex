\chapter{OS API}
The API (Application Programming Interface) for user programs in your operating systems.

\section{Functions}

\subsection{fourchette}
\begin{itemize}
\item{parameters}
\begin{description}
	\item[char *name] program name that will be executed
	\item[int prio] program name that will be executed
\end{description}
\item{return}
\begin{description}
	\item[int pid] pid
\end{description}
\item{description}

fourchette() creates a child process that differs from the parent process only in its PID and PPID.
If success, the PID of the child process is returned in the parent's thread of execution,
and a 0 is returned in the child's thread of execution.
\end{itemize}

\subsection{get\_proc\_info}
\begin{itemize}
\item{parameters}
\begin{description}
	\item[int pid] process pid
	\item[pcb\_info *res] pcb\_info structure
\end{description}
\item{return}
\begin{description}
	\item[int error] error returned if something went wrong 
\end{description}
\item{description}

get\_proc\_info() fill the pcb\_info structure given in parameter with the pcb information.
Only not critical information is given to the user.
\item{structure used : pcb\_info}

This structure is made of:
\begin{itemize}
	\item{pid} process id
	\item{name} process/program name
	\item{pri} process priority
	\item{ls\_supervise} list of supervised processes
	\item{ls\_superviser} list of superviser processes
\end{itemize}
\end{itemize}

\subsection{chgpri}
\begin{itemize}
\item{parameters}
\begin{description}
	\item[int pid] process pid
	\item[int prio] process's new priority
\end{description}
\item{return}
\begin{description}
	\item[int error] error returned if something goes wrong
\end{description}
\item{description}

chgpri changes the priority of the process from the old one to the new priority 'prio'.
\end{itemize}

\subsection{sleep}
\begin{itemize}
\item{parameters}
\begin{description}
	\item[int time] sleep for the specified number of milliseconds
\end{description}
\item{return}
\begin{description}
	\item[void]
\end{description}
\item{description}

sleep()  makes  the  current  process  sleep until 'time' milliseconds seconds have
elapsed.
\end{itemize}

\subsection{wait}
\begin{itemize}
\item{parameters}
\begin{description}
	\item[int *status] sleep for the specified number of milliseconds
\end{description}
\item{return}
\begin{description}
	\item[void]
\end{description}
\item{description}

The wait() function suspends execution of its calling process until
status information is available for a terminated child process.
\end{itemize}

\subsection{sendmsg}
\begin{itemize}
\item{parameters}
\begin{description}
	\item[int msg\_id] message identifier
	\item[int pid\_sender] pid of the sender
	\item[int pid\_receiver] pid of the receiver
	\item[int priority] message priority
	\item[void *user\_data] user data
\end{description}
\item{return}
\begin{description}
	\item[int error] error returned if something went wrong 
\end{description}
\item{description}

The sendmsg() sends a message from a process to another.
User data can be attached to the message using the user\_data pointer.
\end{itemize}

\subsection{recvmsg}
\begin{itemize}
\item{parameters}
\begin{description}
	\item[int timeout] the maximum time a process is waiting for a message. 0 means infinite time.
	\item[int pri] message priority
	\item[msg *message] message to be send
\end{description}
\item{return}
\begin{description}
	\item[int error] error returned if something went wrong 
\end{description}
\item{description}

Wait for a message with a priority equals to pri.
\end{itemize}

\subsection{perror}
\begin{itemize}
\item{parameters}
\begin{description}
	\item[char *msg] error message to print
\end{description}
\item{description}

the perror() function writes the last error that occured followed by a newline, to
the standard output.  If the argument string is non-NULL, this string is prepended
to the message string and separated from it by a colon and space (``: ''); otherwise,
only the error message string is printed.
\end{itemize}

\section{How is the OS API invoked?}
The OS API is invoked by system call.
Implementing system calls requires a control transfer which involves some sort of architecture-specific feature.
A typical way to implement this is to use trap. Interrupts transfer control to the OS so 
software simply needs to set up some register with the system call number they want and execute the software interrupt.

\section{How are programs represented?}
The programs are represented by their address located in the program table.
The program table is the table which associated the program name with the code location in memory.

\section{How can programs that need to communicate locate each other?}
To communicate, programs can send messages between each other. To send a message, a program has to know
the pid of it's target.


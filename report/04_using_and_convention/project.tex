\chapter{Project}

The project name, \textbf{SSIK}, standing for: ``Simply \& Stupidly Implemented Kernel'', is the reversed acronym of \textbf{KISS}. \textbf{KISS} is an acronym for the design principle ``keep it simple and stupid''.

With this project we will try to follow this as a main rule.

\section{Folder hierarchie}

\begin{verbatim}
root
|- bin/
|- build/
|- doc/
|- include/
|- Makefile
|- README
|- report/
|- scripts/
|- src/
   |- kernel/
   |- user/
\end{verbatim}

Description of all the folder:
\begin{description}
  \item[bin] contain the final binarie
  \item[build] contain the object file
  \item[include] contain the API and system library header file
  \item[Makefile] the project Makefile. Support many option, see further
  \item[README] captain obvious prevent me to explain this
  \item[report] contain project report, such as this document
  \item[scripts] script folder, mainly for simisc
  \item[src/kernel] contain the code of the kernel
  \item[src/user] contain the code of all the user software
\end{description}

\section{Makefile}

The make file propose this option:
\begin{description}
  \item[all] build everything
  \item[clean] clean every build
  \item[indent] indent all the source with the indent software
  \item[kernel] only compile the kernel part. But don't link anything
  \item[user] only compile the user part without the system library. But don't link anything
  \item[run] build everything and run the system in simisc
\end{description}


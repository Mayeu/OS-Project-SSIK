\chapter{Good use of git}

\section{General information}

\begin{itemize}
  \item Master is the principal branch
  \item Master should always compile, and should be exempt of obvious bug or malfunction
  \item other may not follow the previous rule
  \item  always work in an other branch than the master. When you want to merge your work with the master ensure that:
	 \begin{itemize}
		\item your work compile
		\item your work is well commented
		\item your master branch is updated with the other master branch (otherwise you can generate conflict)
	 \end{itemize}
  \item  never merge the master branch IN a devellopment branch (i.e. never type \verb+git merge master+ even if it can save your life !)
  \item  if you want to access the last shiny feature of the master branch in your dev branch, use the rebase function (\verb+git rebase master+)
  \item  before pushing your master branch online, pull all other master branch, solve conflict if any, and ensure that everything compile. Afer this, you can push your modification
\end{itemize}

\section{Writing good commit messages}

Here I will massively quote the really good ProGit book (\url{http://progit.org/book/ch5-2.html})

\begin{quote}
  [..] try to make each commit a logically separate changeset. If you can, try to make your changes digestible — don’t code for a whole weekend on five different issues and then submit them all as one massive commit on Monday.
\end{quote}

\begin{quote}
  The last thing to keep in mind is the commit message. Getting in the habit of creating quality commit messages makes using and collaborating with Git a lot easier. As a general rule, your messages should start with a single line that’s no more than about 50 characters and that describes the changeset concisely, followed by a blank line, followed by a more detailed explanation. The Git project requires that the more detailed explanation include your motivation for the change and contrast its implementation with previous behavior — this is a good guideline to follow. It’s also a good idea to use the imperative present tense in these messages. In other words, use commands. Instead of “I added tests for” or “Adding tests for,” use “Add tests for.” Here is a template originally written by Tim Pope at tpope.net:

  \begin{verbatim}
  Short (50 chars or less) summary of changes

  More detailed explanatory text, if necessary.  Wrap it to about 72
  characters or so.  In some contexts, the first line is treated as the
  subject of an email and the rest of the text as the body.  The blank
  line separating the summary from the body is critical (unless you omit
  the body entirely); tools like rebase can get confused if you run the
  two together.

  Further paragraphs come after blank lines.

  - Bullet points are okay, too

  - Typically a hyphen or asterisk is used for the bullet, preceded by a
  single space, with blank lines in between, but conventions vary here


  \end{verbatim}
  If all your commit messages look like this, things will be a lot easier for you and the developers you work with. The Git project has well-formatted commit messages — I encourage you to run git log --no-merges there to see what a nicely formatted project-commit history looks like.
\end{quote}

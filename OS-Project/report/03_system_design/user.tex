\chapter{User programs}
This chapter will describe all the user programs available in our Operating System.

\section{Increment}
Print the sequence 1, 2, 3, ..., n to the console, each number on a new line.

\section{Fibonacci}
Prints the Fibonacci number series, each number on a new line.
Each term of the Fibonacci number series is the sum of the two previous ones.
It starts with the first two numbers 0 and 1 and goes like this: 0, 1, 1, 2, 3, 5 and so on.

\section{Command shell}
Shell that allows user to do some operations on the processes.
It will be able to do at least:
\begin{itemize}
	\item{Start processes}
	\item{Change priority of processes}
	\item{Obtain information about present processes}
	\item{Terminate processes}
	\item{Ouput to the Malta LCD display}
\end{itemize}

\section{Text Scroller}
Scrolls text on the Malta board display.
If the text that must be printed doesn't fit in the Malta LCD display, it will be scrolled
until it has been completely printed, and then the scrolling will restart at the beginning
of the text.
\begin{itemize}
	\item{The process provide for smooth scrolling even on a highly loaded system}
	\item{The scroller is a regular user process with high priority}
	\item{The scroller sleep between updates to the display}
	\item{The scroller start when the operating system starts}
\end{itemize}

If the number of characters to be printed is higher than the number of characters the display
can print, the program will start by printing the first characters, and then will start again,
starting from the second character, and so one until the last charactere of the text is printed
first.

\section{Ring}
User program that demonstrate that message passing communication is working
\begin{itemize}
	\item{A program should start a set of other processes, $P_1$ to $P_n$}
	\item{The processes should be set up in a communications ring, where $P_1$ sends messages to $P_2$, etc. on to $P_n$}
	\item{The demo will send some messages around the ring and show that they visit all processes along the way}
\end{itemize}

\section{Dining philosophers}
User program that demonstrate that process synchronisation is working
The purpose of this program is for the pilosophers to eat, using shared forks.
A philosopher can only eat when he has two forks. Otherwise he will have to wait
for another philosopher to release one of his to start eating.

\section{Process supervision}
User program that demonstrate that process supervision is working.
\begin{itemize}
	\item{Processes can be appointed as supervisors of one or more other processes}
	\item{When a supervised process terminates, the supervisor is notified}
	\item{It is possible to differentiate between controlled and uncontrolled termination,
	  i.e. it is possible for the supervisor to see if a subordinate process has crashed
	  or if it terminated in good order}
\end{itemize}
The demo include a supervisor that restarts its subordinates if they crash.

\section{ps}
User program that displays the currently-running processes.
It will look inside the running list.

\section{malta echo}

User program that print a predefined text on the Malta LDC display.

\section{help}
User program that print the different programms that a user can execute.

\section{Memory management (optional)}

User program that demonstrate that memory management is working.



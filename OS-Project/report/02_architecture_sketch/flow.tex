\chapter{Flow}

\section{System init flow}

\begin{itemize}
  \item print some information (OS name, developer names, system version, etc.)
  \item allocate and initialize all the needed structures for process management
  \item spawn the init process
  \item finish the initialization inside the init process
  \item init spawn the text scroller process
  \item make init the supervisor of the text scroller process
  \item init print that the system is ready
  \item init spawn a shell
\end{itemize}

\section{Process creation}

\begin{itemize}
  \item make a system call (if you are in the user space)
  \item pass the name of the program
  \item check if the program exists
  \item look for available space for the process
  \item allocate the space (if dynamic memory management is set)
  \item initialize the PCB with default values and the program
  \item place the process in the ready queue
  \item return to the caller
\end{itemize}

\section{Message sending}

\begin{itemize}
  \item make a system call (if you are in the user space)
  \item pass the message and the process ID to the syscall
  \item check if the process ID is valid
  \item add the message to the messages list of the process if there is space in the message list.
  \item return to the caller
\end{itemize} 

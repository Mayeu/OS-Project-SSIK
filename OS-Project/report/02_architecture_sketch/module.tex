\chapter{Modules}

\section{General}

\begin{figure}[h!]
  \begin{center}
	 % Generated with LaTeXDraw 2.0.8
% Sun Apr 11 10:20:41 CEST 2010
% \usepackage[usenames,dvipsnames]{pstricks}
% \usepackage{epsfig}
% \usepackage{pst-grad} % For gradients
% \usepackage{pst-plot} % For axes
\scalebox{1} % Change this value to rescale the drawing.
{
\begin{pspicture}(0,-2.1)(9.1,2.1)
\usefont{T1}{ptm}{m}{n}
\rput(4.25,-0.395){\psframebox[linewidth=0.04,framesep=0.2]{Process}}
\usefont{T1}{ptm}{m}{n}
\rput(5.82,-1.395){\psframebox[linewidth=0.04,framesep=0.2]{Scheduler}}
\usefont{T1}{ptm}{m}{n}
\rput(2.59,-0.395){\psframebox[linewidth=0.04,framesep=0.2]{Error}}
\usefont{T1}{ptm}{m}{n}
\rput(7.93,-1.395){\psframebox[linewidth=0.04,framesep=0.2]{Message}}
\usefont{T1}{ptm}{m}{n}
\rput(7.1,-0.395){\psframebox[linewidth=0.04,framesep=0.2]{Memory management}}
\usefont{T1}{ptm}{m}{n}
\rput(1.06,-0.855){\Large Kernel}
\psframe[linewidth=0.04,dimen=outer](9.1,0.3)(0.0,-2.1)
\psframe[linewidth=0.04,dimen=outer](9.1,1.2)(0.0,0.4)
\usefont{T1}{ptm}{m}{n}
\rput(4.81,0.805){System Library}
\usefont{T1}{ptm}{m}{n}
\rput(3.28,-1.395){\psframebox[linewidth=0.04,framesep=0.2]{Kernel Library}}
\usefont{T1}{ptm}{m}{n}
\rput(4.81,1.705){User software}
\psframe[linewidth=0.04,dimen=outer](9.1,2.1)(0.0,1.3)
\psline[linewidth=0.04cm,arrowsize=0.05291667cm 2.0,arrowlength=1.4,arrowinset=0.4]{->}(1.7,0.7)(1.7,0.0)
\psline[linewidth=0.04cm,arrowsize=0.05291667cm 2.0,arrowlength=1.4,arrowinset=0.4]{->}(1.9,0.0)(1.9,0.7)
\psline[linewidth=0.04cm,arrowsize=0.05291667cm 2.0,arrowlength=1.4,arrowinset=0.4]{->}(1.7,1.6)(1.7,0.9)
\psline[linewidth=0.04cm,arrowsize=0.05291667cm 2.0,arrowlength=1.4,arrowinset=0.4]{->}(1.9,0.9)(1.9,1.6)
\end{pspicture} 
}


  \end{center}
  \caption{General Kernel overview}
  \label{fig:gen_kernel}
\end{figure}

\section{Description}
\subsection{Process}

The process module handles process creation, modification, information reading, settings supervised and supervision process etc. Also the management of lists of process (push/pop in the list, search in the lists, sorting, etc.)

\subsection{Scheduler}

The scheduler will be in charge of choosing and picking the next process to run by following the rule of higher priority first and round robin for egal priority process.
This module will use the process module to search and read information about the processes to be able to choose the next process.

\subsection{Error}

This module will provide some facilities to print diagnostics and handle errors (like the perror())

\subsection{Message Handling}

This module will provide function to pass message to different processes. It also provides functions to create/send/read/destroy messages and also manage a list of message.

\subsection{System Library}

The system library will provide some basic and useful functions (e.g. printf(), getc()) through system calls to the kernel. This will also provide an interface to the other modules of the kernel for the user.

This function will only be used by user softwares, kernel software using the kernel library.

\subsection{Kernel Library}

The kernel library will provide, at least, the same basic and useful functions as the system library but ''directly'' (i.e. without syscalls). This function will only be called by kernel software.

\subsection{User software}

This module will provide a bunch of user program such as a shell, curent process information, system information, requirement user program, etc.

\subsection{Kernel}

The ''kernel module'' will be in charge of setting the basic needs of the kernel (default value, list of process, etc.) and launch the init process.

\subsection{Memory management (optional)}

This optional module will provide some dynamic memory management functions. Without this module, all the lists (such as process, message, etc.) will have a static size.

